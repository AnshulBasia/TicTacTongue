\documentclass{article}
\usepackage{authblk}
\usepackage[T1]{fontenc}
\usepackage[utf8]{inputenc}

\title{Recursion, Game-trees and Classes - Making an unbeatable algorithm.}

\author[]{Tushar Rakheja}
\author[]{Shibjash Dutt}
\author[]{Hanit Banga}

\affil[]{\texttt{Instructors @ Endofline Computer Club}}

\renewcommand\Authands{ and }

\date{August 2015}

\begin{document}

\maketitle

\section{Introduction}

In the last session, we talked about functions. We'll start this session by talking about recursive functions. These are functions that refer to themselves, in order to compute what they're meant to. We'll discuss the theory, some classic examples, and some interesting questions that are recursive in nature. \\

\noindent After that, we'll take a slight detour into programming and learn about lists in Python. A list is a data structure \cite{DSinPres_Slide} that stores elements sequentially.\\

\noindent Once we're done with lists, we'll come back to Tic-Tac-Toe, but this time we'll look at it with little to no code. We'll then introduce the tree data structure, and the class programming construct. Finally, we'll put all of it together, and assemble the Earth's Mightiest Zeros. 


\begin{thebibliography}{9}
\bibitem{DSinPres_Slide}
Tushar Rakheja and Shibjash Dutt.
\textit{"CS. What it is, what it isn't, and why you should care."}
\texttt{https://github.com/EndoflineComputerClub/TicTacTongue.}
\\PresentationCS.pptx, slide 11. 
\end{thebibliography}

\end{document}
